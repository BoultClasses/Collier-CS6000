\documentclass[12pt]{article}
\usepackage{palatino}
\usepackage{url}
\usepackage{graphicx}
\usepackage{amsmath}%
\usepackage{amsfonts}%
\usepackage{amssymb}%
\usepackage{setspace}
\usepackage{cite}
\usepackage{hyperref}
\begin{document}
\begin {center}
\section*{CS6000-Journal 1}
By Henry Collier
\\
8/25/2018
\end {center}

\section*{Goals and Introduction}
My primary goal for this course is to learn as much as I can about how to properly conduct effective research.  I hope to learn about the different methods and techniques that others use in their research. I would also like to learn about the different tools that are used in computer science research and how I might incorporate them into my research, to ensure that I am effectively conducting my research. I am in the Security Engineering PhD program and Dr. Chow is my advisor.  My research is different than most of my peers because my research has a significant psychological aspect as I am working to better identify the human behavioral traits that make a person susceptible to social engineering.  As a part of my research, I will be developing a training tool that will incorporate the human behavioral traits identified in the surveys, in order to better secure the human threat. My wife and I are recent empty nesters, as our youngest has just left for college. There are some advantages to being an empty nester, but also some disadvantages as well. For example, I don't have to worry about someone taking a 30 minute shower anymore, but also I don't have the extra set of hands to help with things around the house, so I find myself busier now. 
\begin{figure}[ht!]
\includegraphics[width=30mm]{hcollier.jpg}
\end{figure}
\\
Picture of Henry.
\section*{Git Repo Link}
\url{https://github.com/BoultClasses/Collier-CS6000}
\section*{Tools/Problems/Struggles}
I used TeXworks, part of MikTex 2.9 for this assignment.  I had originally tried to download and install the Mac form of LaTex, but I had difficulty getting it to work correctly, so I downloaded MikTex for my Windows computer.  I had never used any form of LaTex before, and have found this to be more difficult to use than MS Word because of all of the extra steps required to set font or make changes.  I can see how this was probably an importan tool in the early days of computer science research when MS Word or its predacessors were very limited, but at this point I don't see how it is better than the current MS Word or orther "Word" style products. I can input everything in MS Word that I can in LaTex, and do it without having to remember the correct syntax. Perhaps by the end of this course, I will understand why engineers and researchers still use LaTex, but at this point, I don't see it. I overcame my lack of background in LaTex by using a variety of different resources that defined and explained how to incorporate text/photos/formulas within a LaTex document. One that I found very informative was "A Beginners Guide to LaTex" written by David Xiao from Princeton Univeristy and published on September 12, 2005. This resource was very easy to follow and clear in what needed to be done.  
\section*{Paper Review}
\textbf{Paper 1-} Phishing Suspiciousness in Older and Younger Adults: The Role of Executive Functioniong.[1] 
\\
\\
This paper examins how susceptible older adults and younger adults are to phishing through the analysis of the executive functioning ability. The authors used a couple of neuropsychological measures to help determine the susceptibility of the participants.  The Neuropsychological Assessment Battery and the Iowa Gambling Task were used to analyze a relationshiop between poor decision-making and the ventromedial prefrotal cortex dysfunction. 
\\
\\
\textbf {Paper 2-}Who Falls for Phish? A Demographic Analysi of Phisihig Susceptibility and Effectiveness of Interventions. [2]
\\
\\
This paper is an attempt ot study the relationship between domagraphics and phishing susceptiblity through the use of a role-play survey.  The authors are looking to see if there is a definable link between certain demographic characteristics and a person's susceptiblity to being a victim of a phishing attempt. 
\\
\\
\textbf {Paper 3-}Analysis of a Social Engineering Threat to Infomation Security Exacerbated by Vulnerabilities Exposed Through the Inherent Nature of Social Networking Websites.[3] 
\\
\\
This paper analyzes how social engineers are using social networking sites to potentially gain access to the social networking users's employer or other affiliated organizations. 
\\
\\
\textbf {Paper 4-}Curiosity Killed the Organization: A Psychological Comparison Betwen Malicious and Non-malicious Insider and the Insider Threat.[4]
\\
\\
This paper works to better define the traits associated with the malicious insider threat and a non-malicious insider threat. The authors take their research and work to develop methods that can be used to identify if someone is susceptible to becoming either a malicious insider threat or non-malicious insider threat. 
\\
\\
\textbf {Paper 5-}Understanding Insider Threat: A Framework for Characterising Attacks.[5]
\\
\\
This paper addresses a perceived gap in the research into insider threats- a unifying framework that fully characterizes the insider-threat problem space. The authors worked to analyze a variety of insider-threat cases from CMU-CERT, the UK's Centre for Protection of National Infrastructure, and a broad survey of existing academinc and industry research. 
\\
\\
\textbf{Equation-} 
\\
The following is the equation for standard deviation, also known as the amount of variance between a set of values.  
\\
$\sigma =\sqrt{ \frac{1}{N}{\displaystyle\sum_{i=1}^{N} (x_i-\mu)^2}  }   $
\\
This formula is used to calcuate the standard deviation for a set of numbers.  Once the set of numbers has been identified, the first step is to calcualte the mean or simple average and place the value in place of $ \mu$.  Next, subtact the mean ($\mu$) from each value in the set the numbers (represented by $x_i$) and square the results. The next step is to determine the mean for the resulting values represented by the formula ${ \frac{1}{N}{\displaystyle\sum_{i=1}^{N}}}$ where N over sum over e-1 represents the summation of the values and is then multiplied by 1/N, which is the same as dividing the summation by N to get the mean. The final step is to take the square root of  the results from determining the mean that was just calculated.  The final resultant is the standard deviation for the set of numbers. 
\\
\section*{Bibliography}

{[1]} Gavett BE, Zhao R, John SE, Bussell CA, Roberts JR, Yue C (2017) "Phishing suspiciousness in older and younger adults: the role of executive functioning.' PLOS ONE 12(2): e01716220.doi:10.1371/journal.pone.o171620
\\
{[2]} Sjheng S, Holbrook M, Kumaraguru P, Cranor LF, Downs J. (2010) "-Who falls for phish? a demographic analysi of phisihig susceptibility and effectiveness of interventions." Proceedings of the 28th International Conference on Human Factors in Computing Systems, CHI'10
\\
{[3]} Mills D. (2009) "Analsysi of a social engineering threat to information security exacerbated by vulnerabilities exposed throug the inherent nature of social networking websites." InfoSecD'09 September 25-26, 1009 Kenesaw, GA. 
\\
{[4]} Dupuis M, Khadeer S. (2016) "Curiosity killed the organization: a psychological comparison betwen malicious and non-malicious insider and the insider threat." RIIT'16 September 28-October 1, 2016 
\\
{[5]}Nurse J, Buckley O, Legg P, Goldsmith M, Creese S, Wright G, Whitty M. (2014) "Understanding insider threat: a framework for characterising attacks." 2014 IEEE Security and Privacy Workshops. DOI 10.1109/SPW.2014.38

\end{document}