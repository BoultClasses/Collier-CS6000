\documentclass[conference]{IEEEtran}
%\IEEEoverridecommandlockouts
% The preceding line is only needed to identify funding in the first footnote. If that is unneeded, please comment it out.
\usepackage{cite}
\usepackage{amsmath,amssymb,amsfonts}
\usepackage{algorithmic}
\usepackage{graphicx}
\usepackage{textcomp}
\usepackage{xcolor}
\def\BibTeX{{\rm B\kern-.05em{\sc i\kern-.025em b}\kern-.08em
    T\kern-.1667em\lower.7ex\hbox{E}\kern-.125emX}}
\begin{document}

\title{Social Media: A Social Engineers Goldmine*\\
{\footnotesize \textsuperscript{}}
%\thanks{Identify applicable funding agency here. If none, delete this.}
}

\author{\IEEEauthorblockN{Henry Collier}
\IEEEauthorblockA{\textit{Computer Science} \\
\textit{College of Engineering and Applied Science}\\
\textit{University of Colorado, Colorado Springs}\\
W. Berlin, USA \\
hcollier@uccs.edu}

}
\maketitle

\begin{abstract}
Social media is a goldmine for a well-versed social engineer. People put their entire lives on one form of social media or another. For a social engineer, this is an avenue for success as the goldmine found within social media is very data rich.  Whether a social engineer is attempting to “connect” with a potential victim to get close to them in order to steal their identity, or if they are trying to gather enough information to perform a targeted attack the end game is supported in the same manner by the information that the end user posts without realizing the risk associated with their actions.  The purpose of this paper is to demonstrate that a person’s actions on social media puts them at risk to a variety of issues from identity theft to becoming a victim of a malicious threat actor.  
\end{abstract}

\begin{IEEEkeywords}
component, formatting, style, styling, insert
\end{IEEEkeywords}

\section{Introduction}
The problem of social engineering is not new.  In fact, social engineering has been around longer than computers have been.  However, how a social engineer conducts their method of attack has changed and with the ever increasing use of social media, the social engineers game has become one that is easy for them to play. In order to demonstrate how much of a problem social engineering has become, this paper will look at three separate, but connected components that make it easier for a social engineer to be successful. First we will review what social engineering is and how it is conducted, then we will look at social media and how this phenomenon has increased the success rate of social engineers and finally we will look at the human factors that make social media a social engineers goldmine.  

\section{Social Media}

\subsection{Usage}

\textbf {Undergraduate Student Perceptions of Personal Social Media Risk}
This paper looks at results of a study conducted to collect undergraduate student’s perceptions on social media risk. The authors have identified that although students do perceive risk, the risks perceived might not be factual in basis and that further research is needed to clarify if the risks are factual or not. 

\textbf {Data Retrieval from Online Social Network Profiles for Social Engineering Applications}
This paper addresses the issue of automating the extraction of data from social networking sites for use in social engineering attacks. The authors have confirmed that the automated extraction of data for use in social engineering attacks has improved significantly over the last few years. 

\textbf {A Review of Social Media Security Risks and Mitigation Techniques}
This paper reviews the security risks and current mitigation techniques associated with social media in order to develop better mitigation techniques and reduce security risks more effectively. 

\textbf {A Villain’s Guide To Social Media And Web Science}
This paper looks at how villains can use social media to deploy traditional cruelties to great and surprising effect. The authors show that since social media generates a large quantity of information about a user, to include interests and habits, it is possible for a villain to attack both in an intrinsic manner and an extrinsic manner. The authors have shown that although technology can be used by the “virtuous,” it can also be used by individuals with evil intent. 

\textbf {Cyber Threats in Social Networking Websites}
The authors of this paper look into the cyber threats associated with social networking websites.  These threats include the intentional, unintentional, targeted and non-targeted threats that exist due to the nature of social networking sites. The authors of this paper were able to identify several security issues and privacy concerns related to social networking websites and have further identified that advancements within the technology that makes up social networking will bring new security risks with each advancement. 

\textbf {The Threats of Social Networking: Old Wine in New bottles?}
The authors of this paper work to see if the risks associated with social networks are unique and novel.  The authors believe that the risks associated with social network are not unique, but rather enhance long standing, pre-existing attack methods. The authors show that the threats associated with social networking will continue to show up as the technology changes or new communication technologies are presented.  

\textbf {When Social Bots Attack: Modeling Susceptibility of Users in Online Social Networks}
The authors of this paper look at the problem associated with automated bots “acting” as if they were humans and conducting different forms of attack on social networks. The authors were able to determine that susceptible users tended to use Twitter in more of a conversational manner which made them more susceptible to being targeted by a social bot. Furthermore, the authors found that users who were more active were more susceptible to being targeted, which the authors were not surprised to see. 

\subsection{Leakage}

\textbf {Information Leakage through Online Social Networking: Opening the Doorway for Advanced Persistence Threats}
The authors of this paper look at how an employee’s social media habits and behaviors have an effect on an organization and how their actions open the door to advanced persistent threats. The authors have shown that with the increased use of social networks due to mobile devices, it is becoming increasingly more difficult to control the leakage of information that may cause harm to an organization. 

\textbf {Information Revelation and Privacy in Online Social Networks}
The authors of this paper study the patters of information revelation in social networks and what impact this might have on privacy concerns. The authors show users seem to be unconcerned with sharing information about themselves without consideration of the risks associated with doing so. The authors note that in addition to their results being born from actual field data, information revelation can also be determined based on inferred data.  

\subsection{Inference}

\textbf {Inference Attacks Based on Neural Networks in Social Networks}
The authors of this paper look into the issue of privacy related to social networks. In particular, the authors are looking at how semi-private information can be inferred by publicly available information.  This paper highlights a new approach to social engineering that could allow a social engineer to be more successful. The authors acknowledge that this paper is limited in that it only tests data from one social network and one private attribute from that social network. 

\section{Social Engineering}

\textbf {A Taxonomy of Attacks and a Survey of Defence Mechanisms for Semantic Social Engineering Attacks}
The authors of this paper look at the specific version of social engineering known as semantic attacks and identify a taxonomy associated with this style of attack. Furthermore, the authors conduct a survey of the different mitigation methods used to counter these semantic attacks.  The authors conclude that it is difficult to use technical countermeasures against the new/dynamic semantic attacks. The authors acknowledge that although policy control, user education and awareness campaigns are beneficial, a better approach would be to combine both the technical and user management countermeasure into a single combined system. 

\textbf {Advanced Social Engineering Attacks}
The authors of this paper look at the concept of advanced social engineering attacks and how the growing use of collaboration tools has opened new attack vectors for social engineers. The authors believe that in order to develop effective countermeasures to these advanced attacks, there needs to be a detailed understanding of these attack vectors and how they are used. 

\textbf {Cheap and Automated Socio-Technical Attacks based on Social Networking Sites}
The authors of this paper discusses several social-technical attacks, but finish with presenting a novel large-scale socio spam attack based on social networking sites. The authors clearly identify that conducting a large-scale socio spam attack using the vast amount of data that can be collected via a social networking site could have severe and long lasting negative effects. To further the problem, as artificial intelligence and machine learning continues to improve, use of these technologies would increase the success rate of the large-scale spam attack. 

\textbf {Cybersecurity: Risks, Vulnerabilities and Countermeasures to Prevent Social Engineering Attacks.}
The authors of this paper conduct a thorough analysis of what social engineering is and why it is effective. The authors further go on to identify several preventative measures and possible solution to social engineering. 

\textbf {Introducing Cyber Security by Designing Mock Social Engineering Attacks. }
The authors of this paper take on the concept of defending against social engineering attacks by conducting mock attacks whereby the participants can work to learn more about how the attack is conducted, which should allow the participants the opportunity to learn how to better defend against the attacks. The authors identified that by conducting the mock attacks, this reduced the participant’s confidence in identifying attack, which actually resulted in the participant being more suspicious of potential attacks. 

\textbf {Multi-layered Graph-based Model for Social Engineering Vulnerability Assessment}
The authors of this paper address the fact that social engineering attacks rarely are taken into account when vulnerability assessments are completed. The authors describe a multi-layered graph-based model for including social engineering into a vulnerability assessment.  The multi-layered graph-based model allows for the easy capture of the diverse number of channels used by social engineers to attack an organization. 

\textbf {Robot Social Engineering}
The authors of this paper argue that the current state of robotics and artificial intelligence, along with the nature of social engineer creates a situation where it is possible for robots to conduct a “robot social engineering attack.”  Furthermore, authors work to identify potential defenses to a robot social engineering attack. 

\textbf {Social Engineering Attacks on the Knowledge Worker}
The authors of this paper look at how today’s knowledge worker is targeted by social engineers and why they are targeted. The authors detail the modern social engineering attacks on knowledge workers and then define a comprehensive taxonomy to classify social engineer attacks. 

\textbf {The Impact of Social Engineering on Industrial Control System Security}
The authors of this paper use a mean-time-to-compromise metric as a base to conduct a case study and analyze the social engineering attack vectors that could be used to attack Industrial Control Systems(ICS).  As part of the analysis, the authors identify how deep each attack vector could lead an attacker into the ICS system. 

\textbf {The use of Formal Social Engineering Techniques to Identify Weakness During a Computer Vulnerability Competition. }
The authors of this paper conducted a social engineering exercise during the Computer And Network Vulnerability Assessment Simulation (Canvas 2010) to show the vitality of social engineering attacks. The authors have shown that it was very easy to extract information from the participants that could be used for personal benefit. 

\textbf {Two Methodologies for Physical Penetration Testing Using Social Engineering}
The authors of this paper propose two methodologies to conduct penetration testing with social engineering as a component. The authors validated their two methods by conducting a set of penetration test that were conducted over a two-year period. The authors were able to identify that the first method, custodian focused methodology, improves on the environment-focused methodology, but the environment-focused methodology is more reliable.  


\section{Human Factors}
\subsection{Susceptibility}
\textbf {Online Deception in Social Media}
The authors of this paper look at how easy it has become to deceive someone through social media. The ease of deception is due to the behaviors of the users, which increases the susceptibility of the users. The authors show that both individuals and organizations are at risk of being deceived through social media.  Furthermore, the authors show that significant damage that can be inflicted through online social media deception techniques.  

\textbf {Victim Communication Stack (VCS): A Flexible Model to Select the Human Attack Vector}
This authors of this paper look at the category of vulnerabilities known as the Human Attack Vector (HAV). The authors propose a multilayer model called Victim Communication Stack (VCS) which provides several key elements that attackers use to facilitate selection of the HAV. The VCS model is designed to encompass the main characteristics of the HAV, while not limiting itself to being restricted to any particular scenario. The authors have shown that the VCS can be instrumental to attacker to develop better target attacks, but also to defenders by giving them a better way to analyze attacks to their organization. 

\subsection{Habits}

\textbf {Who Falls for Phish? A Demographic Analysis of Phishing Susceptibility and Effectiveness of Interventions}
The authors of this paper work to identify who is more susceptible to phishing and what demographic characteristics determine a person’s susceptibility. Through the use of a role-play survey of 1001 respondents, the authors were able to show that women were more likely to click on a link versus men and that individuals between the ages of 18 and 25 were more likely to fall for the phishing attempt as well. The authors make the assumption that the age group identified is more at risk because most of them have a lower education level than other age groups.  The authors make no assumptions regarding the gender differences found in their research. In fact, the authors do state that age, gender, race and education level do not affect a person’s ability to learn, so that good training plan should reduce the susceptibility in all participants. 

\textbf {Social Media Security Culture}
The authors of this paper look at how incorporating the social media security culture into an organizations overall culture will reduce the risk of an employee being susceptible to social engineer and online deception.  The authors developed a management model that can be used to help managed the organization’s culture and ensure that social media security is incorporated into the culture and managed appropriately to help reduce the threat posture of the organization. The authors identify that there is a problem when it comes to managing social media security due to the duplicity of social media where there is both a private platform and a professional platform that are technically separate, but often times will overlap.  The authors recommend both education and policy as methods to mitigate the risks associated with social media use. 

\section{Classification Model}

\begin{figure}[htbp]
\centerline{\includegraphics[scale= .25]{classification.jpg}}
\caption{Classification model }
\label{fig}
\end{figure}


\nocite{*}
\bibliographystyle{ieeetran}
\bibliography{Bibliography}

\end{document}
